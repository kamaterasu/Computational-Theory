\chapter{Бүлэг 3. Тооцооллын хүндрэл 101: Суурь ойлголтууд, P ба NP}
\label{chap:regular}

Энэ бүлэгт бид тооцооллын асуудлуудын үндсэн ойлголт, өгөгдлийн төлөөлөл, үр ашигтай тооцоолол, асуудлуудын хоорондын үр ашигтай бууруулалт (reduction), нотолгоог үр ашигтайгаар шалгах, P, NP, coNP ангиуд, мөн NP-бүрэн (NP-complete) асуудлын ойлголтыг авч үзнэ. Бид голчлон ангиллын буюу шийдвэрийн (decision) асуудлууд дээр төвлөрнө ба өөр төрлийн асуудлууд ба ангиуд (тооллого, ойролцоолол, байгуулах зэрэг) ч мөн судлагддаг бөгөөд заримыг нь энэ номын дараагийн хэсгүүдэд хэлэлцэнэ.

\section{Сэдэлжүүлэх жишээнүүд}
Дараах гурван шийдвэрийн ангиллын асуудлыг авч үзье. 2-р бүлэгийн адил, ийм төрлийн ангиллын асуудал бүрт бидэнд нэг объектын тайлбар өгөгдөнө, тэгээд тэр нь хүссэн шинжийг агуулж байгаа эсэхийг шийдэх ёстой болно.

\begin{enumerate}
  \item $Ax^2+By+C=0$ хэлбэрийн ямар Диофантын тэгшитгэлүүдийг эерэг бүхэл тооноор шийдэж болох вэ?
  \item 3 хэмжээст маннифолдууд дахь ямар “зангилаа” (knots) нь төрөл $\le g$ бүхий гадаргуугаар хязгаарлагдах вэ?
  \item Ямар хавтгай газрын зураг (планар граф) 3 өнгөөр будагдах боломжтой вэ?
\end{enumerate}

Асуудал (1′) нь 2-р бүлгийн (1) асуудлын хязгаарлалт юм. Асуудал (1) нь шийдэшгүй (undecidable) байсан тул Диофантын тэгшитгэлийн илүү хязгаарлагдсан ангиллуудыг илүү сайн ойлгох нь зүй ёсны хэрэг. Асуудал (2′) нь 2-р бүлгийн (2) асуудлын хоёр талаараа ерөнхийлөлт юм: unknotting асуудал (2) нь маннифолд $R^3$ ба төрөл g=0-ийн тусгай тохиолдлыг авч үздэг. Асуудал (2) нь шийдэгддэг (decidable) байсан бөгөөд түүний ерөнхийлөлт (2′) мөн шийдэгддэг эсэхийг мэдэхийг хүсэж болох юм. Асуудал (3′) нь (3) асуудлын сонирхолтой хувилбар. Бүх газрын зураг 4 өнгөөр будагддаг боловч бүх зураг 3 өнгөөр будагддаггүй; зарим нь чадна, зарим нь чаддаггүй. Тиймээс энэ нь ойлгоход бэрхшээлтэй өөр нэгэн ангиллын асуудал юм.
Ихэнх математикчид эдгээр гурав нь хоорондоо огт холбоогүй асуудлууд гэж үзэх хандлагатай: тус бүр нь өөр өөр чиглэл—тус тусдаа ойлголт, зорилго, хэрэгслүүдтэй альгебр, топологи, комбинаторик харьяалагдана. Гэвч доорх теорем энэ үзлийг буруу байж магадгүйг санал болгож байна.


\begin{theorem}
  (1`), (2`), (3`) асуудлууд нь эквивалент.
\end{theorem}


Цаашлаад, энэ эквивалентийн ойлголт нь байгалийн (natural) бөгөөд бүрэн формаль. Интуитивоор, бид нэг асуудлын талаар олж авсан аливаа ойлголтыг нөгөөгийн ижил төстэй ойлголт болгон энгийнээр “орчуулж” болно. Энэ эквивалентийн формаль утга энэ бүлэгт шат дараалан дэлгэгдэж, 3.9-р хэсэгт формальчлагдана. Түүнд хүрэхийн тулд ийм гайхалтай үр дүнгүүдийг гаргах “хэл” ба “багажлалыг” хөгжүүлэх шаардлагатай.

\vspace{5mm}

\textbf{Төлөөллийн асуудлууд}
Бид (албан бус байдлаар ба жишээгээр) ийм олон талт, төвөгтэй математик объектуудыг хэрхэн хязгаарлагдмал (төгсгөлтэй) аргаар, эцэстээ битүүдийн дараалал болгон дүрслэж болохыг хэлэлцье. Ихэнхдээ нэг объектийг төлөөлөх хэд хэдэн өөр арга байдаг бөгөөд эдгээрийн хооронд хөрвүүлэх нь ихэвчлэн энгийн байдаг. Дараах гурван асуудлын оролтын төлөөллийн талаар ярилцъя.


\textbf{Асуудал (1`)}-ийн хувьд эхлээд $Ax^2+By+C=0$ хэлбэрийн коэффинцинтүүд $A, B, C$-г нь бүхэл тоо байх бүх тэгшитгэлийн цуглуулгыг авч үзье. Ийм тэгшитгэлийг хязгаарлагдмал аргаар төлөөлөх нь илэрхий ба коэффицинтүүдийн гурвал $(A,B,C)$, жишээлбэл тус бүрийг хоёртын бичлэгээр бичнэ. Ийм гурвал өгөгдвөл, холбогдох олон гишүүнт нь эерэг бүхэл $(x,y)$ язгууртай эсэхийг шийдэх шийдвэрийн асуудал үүснэ. Тийм язгуур БИЙ (YES) тохиолддог гурвалуудын дэд олонлоглыг \textbf{$DIO$} гэж тэмдэглэе.


\textbf{Асуудал (2`)}–ийн оролтыг хязгаарлагдмал аргаар төлөөлөх нь арай нарийн боловч байгалийн (natural) байдаг. Оролт нь 3 хэмжээст маннифолд $M$, түүнд суулгасан зангилаа $K$, мөн бүхэл тоо $G$-оос бүрдэнэ. $M$-ийг триангуляциар (тетраэдрүүдийн хязгаарлагдмал цуглуулга ба тэдгээрийн хөршлөлийг зааж) дүрслэж болно. Зангилаа $K$-г өгөгдсөн тетраэдрүүдийн ирмэгүүдийг дагасан хаалттай замаар дүрсэлье. Ийм гурвал $(M,K,G)$ өгөгдвөл, $K$-ээр хязгаарлагдах гадаргуугийн төрөл (genus) нь хамгийн ихдээ $G$ эсэхийг шийднэ. БИЙ хариутай оролтуудын дэд олонлогыг $KNOT$ гэж тэмдэглэе.


\textbf{Асуудал (3′)}–ийн оролтыг хязгаарлагдмал аргаар төлөөлөх нь мөн энгийн биш. Газрын зураг бус, харин улс (орон)-уудыг оройгоор, хил залгаа харилцааг ирмэгээр төлөөлсөн графыг авч үзэцгээе (энэ нь эквивалент: хавтгайн газрын зургийн граф нь түүний хос (dual) зурагтай тэнцүү ойлголт). Графыг (түүнд хавтгай граф байх нь ил тод харагдахаар) дүрслэх нэг гоёмсог боломж нь Fáry-ийн энгийн бөгөөд үзэсгэлэнт теоремийг \cite{Fár48} ашиглах явдал юм (энэ теоремийг бусад нь бие даан нээсэн, олон янзын баталгаа байдаг). Уг теорем нь: бүх хавтгай графыг хавтгайд шулуун ирмэгтэй суулгалтаар (ирмэгүүд огт огтлолцохгүй) дүрсэлж болно. Иймээс оролтыг оройнуудын координатын олонлог $V$ (эдгээр координатуудыг бүр жижиг бүхэл тоонууд байж болно) болон ирмэгүүдийн олонлог $E$ (тус бүр нь $V$-ийн элементүүдийн хос) болгон өгч болно. 3 өнгөөр будаж болох газрын зургийг дүрсэлж буй оролтууд $(V,E)$-ийн дэд олонлогыг $3COL$ гэж авъя.


Ер нь аливаа хязгаарлагдмал объект (бүхэл тоо, бүхэл тоонуудын кортеж, хязгаарлагдмал граф, хязгаарлагдмал комплекс гэх мэт) нь ${0,1}$ цагаан толгой бүхий хоёртын дарааллаар байгалиараа төлөөлөгдөж чадна, алгоритмд оролт өгөхдөө бид ингэж дүрслэнэ. Дээр дурдсанчлан, зангилаа зэрэг тасралтгүй объектуудад ч хязгаарлагдмал тайлбар (опис) байдаг тул ингэж төлөөлж болно\footnote{Алгоритмийн онол нь үргэлжилсэн оролтонд(бодит эсвэл комплекс тоо гэх мэт) зориулагдаж хөгжүүлэгдсэн. Жишээ нь \cite{BCSS98}, \cite{BC06}, Гэхдээ энд яригдахгүй.}. Энд бид объектуудын төлөөлөл давтагдашгүй байх ёстой юу, эсвэл бүр хоёртын дараалал бүр хүчинтэй объект заавал төлөөлөх ёстой юу гэх мэт нарийн асуудлыг хэлэлцэхгүй. Ихэнх “байгалийн” асуудлуудад оролтыг кодлохыг эдгээр нь бодит асуудал бишээр сонгож болдог гэж хэлэхэд хангалттай. Цаашлаад объект ба түүний хоёртын төлөөллийн хооронд “ирж-очих” хөрвүүлэлт нь энгийн бөгөөд үр ашигтай (энэ ойлголтыг доор формаль байдлаар тодорхойлно).


Иймээс $I$-г бүх хязгаарлагдмал урттай хоёртын дарааллын олонлог гэж тэмдэглээд, манай бүх ангиллын асуудлуудын оролтын олонлог гэж үзэцгээе. Үнэхээр, $I$-ийн аливаа дэд олонлог нэг ангиллын асуудлыг тодорхойлно. Энэ хэлээр, $x \in I$ хоёртын дараалал өгөгдвөл, бид түүнийг бүхэл тоонуудын гурвал $(A,B,C)$ гэж тайлбарлаж, холбогдох тэгшитгэл $2DIO$ олонлогд багтаж байна уу гэж асууж болно. Энэ нь (1′) асуудал. Мөн $x$-ийг $(M,K,G)$ маннифолд, зангилаа, бүхэл тоо — гурвал гэж тайлбарлаад $KNOT$ дэд олонлогд орж байна уу гэж асууж болно — энэ нь (2′) асуудал. Ижилээр (3′) дээр ч хийж болно.


\vspace{5mm}


\textbf{Редукц} Теорем 3.1 нь (1′) ба (2′) асуудлыг бодохын хооронд хоёр чиглэлтэй, энгийн хувиргалтууд байдгийг хэлж байна. Тодруулбал, эдгээр хувиргалтыг гүйцэтгэх, үр ашигтайгаар тооцоолж болох $f,h: I \rightarrow I$ функцуудыг өгнө. Үүнд: \\
$(A,B,C) \in 2DIO iff f(A,B,C) \in KNOT$, ба \\
$(M,K,G) \in KNOT iff h(M,K,G) \in 2DIO$ \\


Иймээс эдгээр асуудлуудын нэгийг үр ашигтайгаар (жишээлбэл, полином хугацаанд) шийдэх алгоритм байвал нөгөөд нь даруй ижил төстэй алгоритм гарна. Өөрөөр хэлбэл, хэрвээ бид топологийг хангалттай ойлгож, жишээ нь зангилааны төрөл тодорхойлох асуудлыг шийдэж чаддаг бол, автоматаар тоон онолын ойлголт ч хангалттай болсон бөгөөд квадрати Диофантын эдгээр асуудлыг (ба эсрэгээр нь) шийдэх боломжтой гэсэн үг.


Ийм $f$ ба $h$ хувиргадаг функцийг редукц гэж нэрлэдэг. Редукцын “энгийн”-ийг тооцооллын үүднээс баримтжуулахын тулд тэд үр ашигтайгаар тооцоологдох ёстой гэж шаарддаг.


Ижил маягийн редукцууд газрын зургийн 3-өнгөөр будах асуудал ба нөгөө хоёрын хооронд ч бас байна. Хэрэв графын онолын хангалттай ойлголт бидэнд өгөгдсөн хавтгайн газрын зураг 3-өнгөөр будагдах эсэхийг үр ашигтай тодорхойлох алгоритм өгвөл, түүнтэй ижил төстэй алгоритмууд $KNOT$ ба $2DIO$–д ч дагалдана. Мөн эсрэгээр тэдгээрийн аль нэгийн нь ойлговол 3-өнгөөр будах асуудал адилхан шийдэгдэнэ. Энэ эерэг тайлбар нь гурван асуудлыг адилхан “хүрэгдэхүйц” мэт харагдуулна. Гэвч нөгөө тал нь: тэд мөн адилхан хүнд хэрэв аль нэгэнд нь ийм үр ашигтай ангилах алгоритм байхгүй бол нөгөө хоёрт нь ч байхгүй гэсэн үг. Үнэндээ өнөөдөр эдгээр эквивалентийн талаар илүү сайн ойлголттой болсон учир “хоёр дахь” тайлбар илүү магадлалтай: эдгээр асуудлуудыг ойлгох нь бүхэлдээ хэцүү.


Энэ материалыг ангидаа эсвэл урьдчилан таамаглаагүй сонсогчидод лекцээр тайлбарлахад, ийм алс холын асуудлуудын хоорондын гэнэтийн, хүчтэй холбоонд хүмүүс хэрхэн гайхшран хүлээж авдгийг харах нь сонирхолтой. Танд ч бас ийм нөлөө үзүүлээсэй гэж найдаж байна. Харин одоо энэ “нууцыг” тайлж, эдгээр холбоосын эх сурвалжийг тайлбарлая. Ингээд эхэлье.


\section{Үр ашигтай тооцоолол ба P анги}


Үр ашигтай алгоритмууд нь үйлдвэрлэл, эдийн засгийн улам бүр өсөн нэмэгдэж буй хэсгийг, түүнчлэн таны өдөр тутмын амьдралыг хөдөлгөдөг хөдөлгүүр юм. Эдгээр “сувд” нь таны өдөр бүр ашигладаг ихэнх төхөөрөмж, хэрэглээний програмуудад шингэсэн байдаг. Энэ хэсэгт бид үр ашигтай тооцооллын математик ойлголт, полиномиаль хугацааны алгоритмыг абстракчилж, шалтгаан, жишээг үзнэ.


Одоо бид асимптотик нарийн төвөгтэй байдалд төвлөрнө. Жишээлбэл, бид $2^67 − 1$ тоог (Мерсенн энэ талаар хэт их анхаарсан шиг) задлахад хэдий хугацаа зарцуулах, эсвэл бүх 67-бит тоог задлах хугацаа гэхээсээ илүү оролтын урт n-ийн функц байдлаар n-бит тоонуудыг задлахын асимптотик зан төлөвт анхаарна. Асимптотик харах өнцөг нь тооцооллын хүндрэлийн онолд салшгүй бөгөөд энэ номоос харахад төгсгөлийн, яг-точны шинжилгээгээр бүдгэрэх бүтэц, хэв маягийг илрүүлдгийг үзнэ. Оролтын хэмжээнээс хамаарах байдал нь тооцоололлын онолд (Computability theory) байдаггүй. Тэнд алгоритм нь зүгээр л эцсийн хугацаанд зогсох ёстой. Гэсэн ч эдгээр салбаруудын аргачлалын ихэнх нь тооцооллын нарийн төвөгтэй байдал руу “импортлогдсон” байдаг асуудлын ангиллууд, асуудлууд хоорондын буулгалт (reduction), бүрэн (complete) асуудлууд бүгдийг нь бид цаашид үзнэ.


Тодорхой (өгөгдсөн) асуудлын хувьд \textit{үр ашигтай тооцоолол} гэж оролтын урт n бүхий аливаа оролт дээрх ажиллах хугацаа нь n-ийн полиномиаль функцээр хязгаарлагддаг алгоритмыг ойлгоно.


$I$-г бүх боломжит урттай хоёртын дарааллуудын олонлог гэж тэмдэглэе. $I_n$ нь урт нь $n$ байх $I$ доторх бүх хоёртын дарааллыг, өөрөөр хэлбэл $I_n = {0,1}^n$-ийг илэрхийлнэ.


\begin{definition}
  (P Анги). $f : I \rightarrow I$ функц нь дараах нөхцөлийг хангавал P ангид багтана: f-ийг тооцоолох алгоритм болон эерэг тогтмолууд $A$, $c$ оршин байх ба дурын $n$, дурын $x \in I_n$ дээр уг алгоритм хамгийн ихдээ $An^c$ алхам (өөрөөр хэлбэл, анхан шатны үйлдлүүд) хийж $f(x)-ийг$ бодно.
\end{definition}


Энэхүү тодорхойлолт нь ялангуяа гаралт нь {0,1} байх $Бүүлийн (Boolean) функцууд$ ангилах (шийдвэрлэх) асуудлуудыг илэрхийлэгчид—д хамаарна. Бид тэмдэглэгээг бага зэрэг “зөрчин” заримдаа $P$-ийг зөвхөн эдгээр ангилах асуудлуудыг агуулсан анги мэтээр үзэх нь бий. Урт гаралттай функцийг гаралтын бүр битэд харгалзах Бүүлийн функцуудын дараалал гэж үзэж болно.


Полиномиаль өсөлтийг (brute-force экспоненциал өсөлттэй) эсрэгцүүлсэн энэ чухал тодорхойлолтыг 1960-аад оны төгсгөлд Кобхам \cite{Cob65}, Эдмондс cite{Edm65b, Edm66, Edm67a}, Рабин \cite{Rab67} нар дэвшүүлсэн. Өөр өөр чиглэл, зорилгоос ирсэн эдгээр судлаачид үр ашигтай алгоритмыг зүгээр л эцэст нь зогсдог алгоритмаас албан ёсоор ялган зааглахыг оролдсон. Ялангуяа Эдмондсын өгүүллүүд нь байгалийн оновчлолын зарим асуудлуудад ухаалаг полиномиаль хугацааны алгоритмуудыг санал болгосон байдаг. Мэдээж хэрэг, компьютерийн эринээс өмнө ч сонирхолтой полиномиаль хугацааны алгоритмууд олныг нээсэн. Гараар тооцоолох үр ашигтай арга шаардсан математикчид тэдгээрийг бүтээжээ. Хамгийн эртний, алдартай жишээ нь, I бүлэгт дурдах Евклидийн ХИЕХ (GCD) алгоритм бөгөөд энэ нь бүхэл тоонуудын хамгийн их ерөнхий хуваагчийг олоход анх тоог задлах шаардлагыг тойрч гарахаар зохиогдсон юм.


P-ийг үр ашигтай тооцоолох боломжтой функцуудын анги болгохоор сонгохдоо хоёр томоохон сонголт зайлшгүй хийнэ. Энэ нь ихээхэн хэлэлцүүлэгддэг, мөн тайлбар шаардана. Нэгдүгээрт, оролтын уртаас хамаарах хугацааны дээд хязгаарыг \textit{полиномиаль} гэж сонгосон явдал. Хоёрдугаарт, энэ хугацааны хязгаарлалт бүх оролтод хүчинтэй байх ёстой гэсэн хамгийн \textit{муу тохиолдлын (worst-case)} шаардлага. Эдгээр хоёр сонголтын үндэслэл, ач холбогдлыг бид доор хэлэлцэнэ. Гэхдээ эдгээр нь сахилга баттай онол биш гэдгийг онцлох нь зүйтэй: тооцооллын хүндрэлийн онолд дээрх сонголтуудад олон өөр хувилбаруудыг авч үзэж, судалж ирсэн. Үүнд полиномиалиас өөр, илүү нарийн ялгавартай үр ашигтай байдлын хязгаарууд, мөн хамгийн муу тохиолдлыг орлох дундаж-тохиолдлын болон оролтоос хамаарах янз бүрийн хэмжүүрүүд орно. Эдгээрийн заримыг номын цаашдын хэсгүүдэд авч үзнэ. Гэсэн ч дээрх анхны сонголтууд нь тооцооллын нарийн төвөгтэй байдлын эхэн үед асар чухал байж, энэ салбарын үзэсгэлэнт бүтэц-ийг илрүүлж, бат бөх суурийг тавьж, арга зүйг тогтоон, дараа дараагийн илүү нарийн, олон талт хувилбаруудын судалгааг чиглүүлсэн билээ.


\subsection{Яагаад полиномиаль вэ?}

