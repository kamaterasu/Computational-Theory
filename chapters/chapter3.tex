\chapter{Бүлэг 3. Тооцооллын хүндрэл 101: Суурь ойлголтууд, P ба NP}
\label{chap:regular}

Энэ бүлэгт бид тооцооллын асуудлуудын үндсэн ойлголт, өгөгдлийн төлөөлөл, үр ашигтай тооцоолол, асуудлуудын хоорондын үр ашигтай бууруулалт (reduction), нотолгоог үр ашигтайгаар шалгах, P, NP, coNP ангиуд, мөн NP-бүрэн (NP-complete) асуудлын ойлголтыг авч үзнэ. Бид голчлон ангиллын буюу шийдвэрийн (decision) асуудлууд дээр төвлөрнө ба өөр төрлийн асуудлууд ба ангиуд (тооллого, ойролцоолол, байгуулах зэрэг) ч мөн судлагддаг бөгөөд заримыг нь энэ номын дараагийн хэсгүүдэд хэлэлцэнэ.

\section{Сэдэлжүүлэх жишээнүүд}
Дараах гурван шийдвэрийн ангиллын асуудлыг авч үзье. 2-р бүлэгийн адил, ийм төрлийн ангиллын асуудал бүрт бидэнд нэг объектын тайлбар өгөгдөнө, тэгээд тэр нь хүссэн шинжийг агуулж байгаа эсэхийг шийдэх ёстой болно.

\begin{enumerate}
  \item $Ax^2+By+C=0$ хэлбэрийн ямар Диофантын тэгшитгэлүүдийг эерэг бүхэл тооноор шийдэж болох вэ?
  \item 3 хэмжээст маннифолдууд дахь ямар “зангилаа” (knots) нь төрөл $\le g$ бүхий гадаргуугаар хязгаарлагдах вэ?
  \item Ямар хавтгай газрын зураг (планар граф) 3 өнгөөр будагдах боломжтой вэ?
\end{enumerate}

Асуудал (1′) нь 2-р бүлгийн (1) асуудлын хязгаарлалт юм. Асуудал (1) нь шийдэшгүй (undecidable) байсан тул Диофантын тэгшитгэлийн илүү хязгаарлагдсан ангиллуудыг илүү сайн ойлгох нь зүй ёсны хэрэг. Асуудал (2′) нь 2-р бүлгийн (2) асуудлын хоёр талаараа ерөнхийлөлт юм: unknotting асуудал (2) нь маннифолд $R^3$ ба төрөл g=0-ийн тусгай тохиолдлыг авч үздэг. Асуудал (2) нь шийдэгддэг (decidable) байсан бөгөөд түүний ерөнхийлөлт (2′) мөн шийдэгддэг эсэхийг мэдэхийг хүсэж болох юм. Асуудал (3′) нь (3) асуудлын сонирхолтой хувилбар. Бүх газрын зураг 4 өнгөөр будагддаг боловч бүх зураг 3 өнгөөр будагддаггүй; зарим нь чадна, зарим нь чаддаггүй. Тиймээс энэ нь ойлгоход бэрхшээлтэй өөр нэгэн ангиллын асуудал юм.
Ихэнх математикчид эдгээр гурав нь хоорондоо огт холбоогүй асуудлууд гэж үзэх хандлагатай: тус бүр нь өөр өөр чиглэл—тус тусдаа ойлголт, зорилго, хэрэгслүүдтэй альгебр, топологи, комбинаторик харьяалагдана. Гэвч доорх теорем энэ үзлийг буруу байж магадгүйг санал болгож байна.


\begin{theorem}
  (1`), (2`), (3`) асуудлууд нь эквивалент.
\end{theorem}


Цаашлаад, энэ эквивалентийн ойлголт нь байгалийн (natural) бөгөөд бүрэн формаль. Интуитивоор, бид нэг асуудлын талаар олж авсан аливаа ойлголтыг нөгөөгийн ижил төстэй ойлголт болгон энгийнээр “орчуулж” болно. Энэ эквивалентийн формаль утга энэ бүлэгт шат дараалан дэлгэгдэж, 3.9-р хэсэгт формальчлагдана. Түүнд хүрэхийн тулд ийм гайхалтай үр дүнгүүдийг гаргах “хэл” ба “багажлалыг” хөгжүүлэх шаардлагатай.

\vspace{5mm}

\textbf{Төлөөллийн асуудлууд}
Бид (албан бус байдлаар ба жишээгээр) ийм олон талт, төвөгтэй математик объектуудыг хэрхэн хязгаарлагдмал (төгсгөлтэй) аргаар, эцэстээ битүүдийн дараалал болгон дүрслэж болохыг хэлэлцье. Ихэнхдээ нэг объектийг төлөөлөх хэд хэдэн өөр арга байдаг бөгөөд эдгээрийн хооронд хөрвүүлэх нь ихэвчлэн энгийн байдаг. Дараах гурван асуудлын оролтын төлөөллийн талаар ярилцъя.


\textbf{Асуудал (1`)}-ийн хувьд эхлээд $Ax^2+By+C=0$ хэлбэрийн коэффинцинтүүд $A, B, C$-г нь бүхэл тоо байх бүх тэгшитгэлийн цуглуулгыг авч үзье. Ийм тэгшитгэлийг хязгаарлагдмал аргаар төлөөлөх нь илэрхий ба коэффицинтүүдийн гурвал $(A,B,C)$, жишээлбэл тус бүрийг хоёртын бичлэгээр бичнэ. Ийм гурвал өгөгдвөл, холбогдох олон гишүүнт нь эерэг бүхэл $(x,y)$ язгууртай эсэхийг шийдэх шийдвэрийн асуудал үүснэ. Тийм язгуур БИЙ (YES) тохиолддог гурвалуудын дэд олонлоглыг \textbf{$DIO$} гэж тэмдэглэе.


\textbf{Асуудал (2`)}–ийн оролтыг хязгаарлагдмал аргаар төлөөлөх нь арай нарийн боловч байгалийн (natural) байдаг. Оролт нь 3 хэмжээст маннифолд $M$, түүнд суулгасан зангилаа $K$, мөн бүхэл тоо $G$-оос бүрдэнэ. $M$-ийг триангуляциар (тетраэдрүүдийн хязгаарлагдмал цуглуулга ба тэдгээрийн хөршлөлийг зааж) дүрслэж болно. Зангилаа $K$-г өгөгдсөн тетраэдрүүдийн ирмэгүүдийг дагасан хаалттай замаар дүрсэлье. Ийм гурвал $(M,K,G)$ өгөгдвөл, $K$-ээр хязгаарлагдах гадаргуугийн төрөл (genus) нь хамгийн ихдээ $G$ эсэхийг шийднэ. БИЙ хариутай оролтуудын дэд олонлогыг $KNOT$ гэж тэмдэглэе.


\textbf{Асуудал (3′)}–ийн оролтыг хязгаарлагдмал аргаар төлөөлөх нь мөн энгийн биш. Газрын зураг бус, харин улс (орон)-уудыг оройгоор, хил залгаа харилцааг ирмэгээр төлөөлсөн графыг авч үзэцгээе (энэ нь эквивалент: хавтгайн газрын зургийн граф нь түүний хос (dual) зурагтай тэнцүү ойлголт). Графыг (түүнд хавтгай граф байх нь ил тод харагдахаар) дүрслэх нэг гоёмсог боломж нь Fáry-ийн энгийн бөгөөд үзэсгэлэнт теоремийг \cite{Fár48} ашиглах явдал юм (энэ теоремийг бусад нь бие даан нээсэн, олон янзын баталгаа байдаг). Уг теорем нь: бүх хавтгай графыг хавтгайд шулуун ирмэгтэй суулгалтаар (ирмэгүүд огт огтлолцохгүй) дүрсэлж болно. Иймээс оролтыг оройнуудын координатын олонлог $V$ (эдгээр координатуудыг бүр жижиг бүхэл тоонууд байж болно) болон ирмэгүүдийн олонлог $E$ (тус бүр нь $V$-ийн элементүүдийн хос) болгон өгч болно. 3 өнгөөр будаж болох газрын зургийг дүрсэлж буй оролтууд $(V,E)$-ийн дэд олонлогыг $3COL$ гэж авъя.


Ер нь аливаа хязгаарлагдмал объект (бүхэл тоо, бүхэл тоонуудын кортеж, хязгаарлагдмал граф, хязгаарлагдмал комплекс гэх мэт) нь ${0,1}$ цагаан толгой бүхий хоёртын дарааллаар байгалиараа төлөөлөгдөж чадна, алгоритмд оролт өгөхдөө бид ингэж дүрслэнэ. Дээр дурдсанчлан, зангилаа зэрэг тасралтгүй объектуудад ч хязгаарлагдмал тайлбар (опис) байдаг тул ингэж төлөөлж болно\footnote{Алгоритмийн онол нь үргэлжилсэн оролтонд(бодит эсвэл комплекс тоо гэх мэт) зориулагдаж хөгжүүлэгдсэн. Жишээ нь \cite{BCSS98}, \cite{BC06}, Гэхдээ энд яригдахгүй.}. Энд бид объектуудын төлөөлөл давтагдашгүй байх ёстой юу, эсвэл бүр хоёртын дараалал бүр хүчинтэй объект заавал төлөөлөх ёстой юу гэх мэт нарийн асуудлыг хэлэлцэхгүй. Ихэнх “байгалийн” асуудлуудад оролтыг кодлохыг эдгээр нь бодит асуудал бишээр сонгож болдог гэж хэлэхэд хангалттай. Цаашлаад объект ба түүний хоёртын төлөөллийн хооронд “ирж-очих” хөрвүүлэлт нь энгийн бөгөөд үр ашигтай (энэ ойлголтыг доор формаль байдлаар тодорхойлно).


Иймээс $I$-г бүх хязгаарлагдмал урттай хоёртын дарааллын олонлог гэж тэмдэглээд, манай бүх ангиллын асуудлуудын оролтын олонлог гэж үзэцгээе. Үнэхээр, $I$-ийн аливаа дэд олонлог нэг ангиллын асуудлыг тодорхойлно. Энэ хэлээр, $x \in I$ хоёртын дараалал өгөгдвөл, бид түүнийг бүхэл тоонуудын гурвал $(A,B,C)$ гэж тайлбарлаж, холбогдох тэгшитгэл $2DIO$ олонлогд багтаж байна уу гэж асууж болно. Энэ нь (1′) асуудал. Мөн $x$-ийг $(M,K,G)$ маннифолд, зангилаа, бүхэл тоо — гурвал гэж тайлбарлаад $KNOT$ дэд олонлогд орж байна уу гэж асууж болно — энэ нь (2′) асуудал. Ижилээр (3′) дээр ч хийж болно.


\vspace{5mm}


\textbf{Редукц} Теорем 3.1 нь (1′) ба (2′) асуудлыг бодохын хооронд хоёр чиглэлтэй, энгийн хувиргалтууд байдгийг хэлж байна. Тодруулбал, эдгээр хувиргалтыг гүйцэтгэх, үр ашигтайгаар тооцоолж болох $f,h: I \rightarrow I$ функцуудыг өгнө. Үүнд: \\
$(A,B,C) \in 2DIO iff f(A,B,C) \in KNOT$, ба \\
$(M,K,G) \in KNOT iff h(M,K,G) \in 2DIO$ \\


Иймээс эдгээр асуудлуудын нэгийг үр ашигтайгаар (жишээлбэл, полином хугацаанд) шийдэх алгоритм байвал нөгөөд нь даруй ижил төстэй алгоритм гарна. Өөрөөр хэлбэл, хэрвээ бид топологийг хангалттай ойлгож, жишээ нь зангилааны төрөл тодорхойлох асуудлыг шийдэж чаддаг бол, автоматаар тоон онолын ойлголт ч хангалттай болсон бөгөөд квадрати Диофантын эдгээр асуудлыг (ба эсрэгээр нь) шийдэх боломжтой гэсэн үг.


Ийм $f$ ба $h$ хувиргадаг функцийг редукц гэж нэрлэдэг. Редукцын “энгийн”-ийг тооцооллын үүднээс баримтжуулахын тулд тэд үр ашигтайгаар тооцоологдох ёстой гэж шаарддаг.


Ижил маягийн редукцууд газрын зургийн 3-өнгөөр будах асуудал ба нөгөө хоёрын хооронд ч бас байна. Хэрэв графын онолын хангалттай ойлголт бидэнд өгөгдсөн хавтгайн газрын зураг 3-өнгөөр будагдах эсэхийг үр ашигтай тодорхойлох алгоритм өгвөл, түүнтэй ижил төстэй алгоритмууд $KNOT$ ба $2DIO$–д ч дагалдана. Мөн эсрэгээр тэдгээрийн аль нэгийн нь ойлговол 3-өнгөөр будах асуудал адилхан шийдэгдэнэ. Энэ эерэг тайлбар нь гурван асуудлыг адилхан “хүрэгдэхүйц” мэт харагдуулна. Гэвч нөгөө тал нь: тэд мөн адилхан хүнд хэрэв аль нэгэнд нь ийм үр ашигтай ангилах алгоритм байхгүй бол нөгөө хоёрт нь ч байхгүй гэсэн үг. Үнэндээ өнөөдөр эдгээр эквивалентийн талаар илүү сайн ойлголттой болсон учир “хоёр дахь” тайлбар илүү магадлалтай: эдгээр асуудлуудыг ойлгох нь бүхэлдээ хэцүү.


Энэ материалыг ангидаа эсвэл урьдчилан таамаглаагүй сонсогчидод лекцээр тайлбарлахад, ийм алс холын асуудлуудын хоорондын гэнэтийн, хүчтэй холбоонд хүмүүс хэрхэн гайхшран хүлээж авдгийг харах нь сонирхолтой. Танд ч бас ийм нөлөө үзүүлээсэй гэж найдаж байна. Харин одоо энэ “нууцыг” тайлж, эдгээр холбоосын эх сурвалжийг тайлбарлая. Ингээд эхэлье.


\section{Үр ашигтай тооцоолол ба P анги}


Үр ашигтай алгоритмууд нь үйлдвэрлэл, эдийн засгийн улам бүр өсөн нэмэгдэж буй хэсгийг, түүнчлэн таны өдөр тутмын амьдралыг хөдөлгөдөг хөдөлгүүр юм. Эдгээр “сувд” нь таны өдөр бүр ашигладаг ихэнх төхөөрөмж, хэрэглээний програмуудад шингэсэн байдаг. Энэ хэсэгт бид үр ашигтай тооцооллын математик ойлголт, полиномиаль хугацааны алгоритмыг абстракчилж, шалтгаан, жишээг үзнэ.


Одоо бид асимптотик нарийн төвөгтэй байдалд төвлөрнө. Жишээлбэл, бид $2^67 − 1$ тоог (Мерсенн энэ талаар хэт их анхаарсан шиг) задлахад хэдий хугацаа зарцуулах, эсвэл бүх 67-бит тоог задлах хугацаа гэхээсээ илүү оролтын урт n-ийн функц байдлаар n-бит тоонуудыг задлахын асимптотик зан төлөвт анхаарна. Асимптотик харах өнцөг нь тооцооллын хүндрэлийн онолд салшгүй бөгөөд энэ номоос харахад төгсгөлийн, яг-точны шинжилгээгээр бүдгэрэх бүтэц, хэв маягийг илрүүлдгийг үзнэ. Оролтын хэмжээнээс хамаарах байдал нь тооцоололлын онолд (Computability theory) байдаггүй. Тэнд алгоритм нь зүгээр л эцсийн хугацаанд зогсох ёстой. Гэсэн ч эдгээр салбаруудын аргачлалын ихэнх нь тооцооллын нарийн төвөгтэй байдал руу “импортлогдсон” байдаг асуудлын ангиллууд, асуудлууд хоорондын буулгалт (reduction), бүрэн (complete) асуудлууд бүгдийг нь бид цаашид үзнэ.


Тодорхой (өгөгдсөн) асуудлын хувьд \textit{үр ашигтай тооцоолол} гэж оролтын урт n бүхий аливаа оролт дээрх ажиллах хугацаа нь n-ийн полиномиаль функцээр хязгаарлагддаг алгоритмыг ойлгоно.


$I$-г бүх боломжит урттай хоёртын дарааллуудын олонлог гэж тэмдэглэе. $I_n$ нь урт нь $n$ байх $I$ доторх бүх хоёртын дарааллыг, өөрөөр хэлбэл $I_n = {0,1}^n$-ийг илэрхийлнэ.


\begin{definition}
  (P Анги). $f : I \rightarrow I$ функц нь дараах нөхцөлийг хангавал P ангид багтана: f-ийг тооцоолох алгоритм болон эерэг тогтмолууд $A$, $c$ оршин байх ба дурын $n$, дурын $x \in I_n$ дээр уг алгоритм хамгийн ихдээ $An^c$ алхам (өөрөөр хэлбэл, анхан шатны үйлдлүүд) хийж $f(x)-ийг$ бодно.
\end{definition}


Энэхүү тодорхойлолт нь ялангуяа гаралт нь {0,1} байх $Бүүлийн (Boolean) функцууд$ ангилах (шийдвэрлэх) асуудлуудыг илэрхийлэгчид—д хамаарна. Бид тэмдэглэгээг бага зэрэг “зөрчин” заримдаа $P$-ийг зөвхөн эдгээр ангилах асуудлуудыг агуулсан анги мэтээр үзэх нь бий. Урт гаралттай функцийг гаралтын бүр битэд харгалзах Бүүлийн функцуудын дараалал гэж үзэж болно.


Полиномиаль өсөлтийг (brute-force экспоненциал өсөлттэй) эсрэгцүүлсэн энэ чухал тодорхойлолтыг 1960-аад оны төгсгөлд Кобхам \cite{Cob65}, Эдмондс \cite{Edm65b, Edm66, Edm67a}, Рабин \cite{Rab67} нар дэвшүүлсэн. Өөр өөр чиглэл, зорилгоос ирсэн эдгээр судлаачид үр ашигтай алгоритмыг зүгээр л эцэст нь зогсдог алгоритмаас албан ёсоор ялган зааглахыг оролдсон. Ялангуяа Эдмондсын өгүүллүүд нь байгалийн оновчлолын зарим асуудлуудад ухаалаг полиномиаль хугацааны алгоритмуудыг санал болгосон байдаг. Мэдээж хэрэг, компьютерийн эринээс өмнө ч сонирхолтой полиномиаль хугацааны алгоритмууд олныг нээсэн. Гараар тооцоолох үр ашигтай арга шаардсан математикчид тэдгээрийг бүтээжээ. Хамгийн эртний, алдартай жишээ нь, I бүлэгт дурдах Евклидийн ХИЕХ (GCD) алгоритм бөгөөд энэ нь бүхэл тоонуудын хамгийн их ерөнхий хуваагчийг олоход анх тоог задлах шаардлагыг тойрч гарахаар зохиогдсон юм.


P-ийг үр ашигтай тооцоолох боломжтой функцуудын анги болгохоор сонгохдоо хоёр томоохон сонголт зайлшгүй хийнэ. Энэ нь ихээхэн хэлэлцүүлэгддэг, мөн тайлбар шаардана. Нэгдүгээрт, оролтын уртаас хамаарах хугацааны дээд хязгаарыг \textit{полиномиаль} гэж сонгосон явдал. Хоёрдугаарт, энэ хугацааны хязгаарлалт бүх оролтод хүчинтэй байх ёстой гэсэн хамгийн \textit{муу тохиолдлын (worst-case)} шаардлага. Эдгээр хоёр сонголтын үндэслэл, ач холбогдлыг бид доор хэлэлцэнэ. Гэхдээ эдгээр нь сахилга баттай онол биш гэдгийг онцлох нь зүйтэй: тооцооллын хүндрэлийн онолд дээрх сонголтуудад олон өөр хувилбаруудыг авч үзэж, судалж ирсэн. Үүнд полиномиалиас өөр, илүү нарийн ялгавартай үр ашигтай байдлын хязгаарууд, мөн хамгийн муу тохиолдлыг орлох дундаж-тохиолдлын болон оролтоос хамаарах янз бүрийн хэмжүүрүүд орно. Эдгээрийн заримыг номын цаашдын хэсгүүдэд авч үзнэ. Гэсэн ч дээрх анхны сонголтууд нь тооцооллын нарийн төвөгтэй байдлын эхэн үед асар чухал байж, энэ салбарын үзэсгэлэнт бүтэц-ийг илрүүлж, бат бөх суурийг тавьж, арга зүйг тогтоон, дараа дараагийн илүү нарийн, олон талт хувилбаруудын судалгааг чиглүүлсэн билээ.


\subsection{Яагаад полиномиаль вэ?}


Үр ашигтай тооцооллыг төлөөлөхөд полиномиаль хугацаа (polynomial time)-г сонгосон нь дураараа мэт санагдаж болох ч энэ сонголт олон талаас нь авч үзвэл цаг хугацааны хувьд их зөв юм. Гол үндэслэлүүдээс заримыг нь энд жагсаая.


Полиномууд нь “удаан өсөлттэй” функцуудын жишээ юм. Полиномууд нэмэх, үржих, композици (composition)-ийн дор хаалттай байдаг нь хоёр программыг дараалан хэрэглэх, нэгийг нь нөгөөгийн дэд програм (subroutine) болгон ашиглах зэрэг натурал програмчлалын дадалд үр ашгийн ойлголтыг хадгалж өгдөг. Энэ сонголт нь тооцооллын загварыг яг таг тодорхойлж өгөх шаардлагыг арилгана (жишээлбэл, арифметикийн үйлдлүүдийг ганц орон дээр хийх үү, эсвэл дурын бүхэл тоон дээр хийхийг зөвшөөрөх үү гэдэг нь хамаагүй; учир нь урт хэлбэрийн нэмэх, хасах, үржих, хуваахын энгийн полиномиаль хугацааны алгоритмууд бага сургуульд заадаг). Үүнтэй адилаар өгөгдлийн дүрслэл дээр санаа зовох хэрэггүй: хязгаарлагдмал объектуудын олонлогийн бараг аливаа хоёр байгаллаг дүрслэл хооронд үр ашигтай хөрвүүлэх боломжтой.


Практик талаас нь харахад ажиллах хугацаа нь, жишээ нь, $n²$ байх нь $n^100$-оос хамаагүй илүү, мэдээж шулуун шугаман (linear) хугацаа бүр сайн. Үнэндээ полиномиаль хугацааны тогтмол үржүүлэгч хүртэл бодит амьдрал дахь боломжит хэрэгжилтэд шийдвэрлэх ач холбогдолтой байж болно. Гэсэн ч “жижиг тоонуудын хууль” гэдэг зүйл ажиглагддаг ба байгалийн асуудлуудад олдсон полиномиаль хугацааны алгоритмуудын зэрэг 3–4-өөс дээш байх нь тун ховор (алгоритм нээгдсэн мөчид эхний зэрэг нь 30, 40 байсан тохиолдол байдаг ч дараа нь эрс сайжирдаг). Харин одоог хүртэл үр ашигтай алгоритмд эсэргүүцэлтэй олон чухал натурал асуудлуудыг экспоненциал хугацаанаас хурдан бодож чадахгүй байна (энэ нь мэдээж, жижиг оролтод ч практикт огт тохирохгүй). Энэ экспоненциал ангал нь P ангийг тодорхойлох урам зоригийг бий болгодог. Ийм асуудлуудын нарийн төвөгтэй байдлыг экспоненциалаас (ямар ч байсан) полиномиаль болгож бууруулна гэдэг нь асар том, шинэ арга барил шаардсан ойлголтын дэвшил юм.


\subsection{Яагаад хамгийн муу тохиолдол вэ?}


P ангийг шүүмжлэх өөр нэг өнцөг нь уртынхаа хувьд n-тэй тэнцүү бүх оролтыг poly(n)-хугацаанд бодож чадвал тухайн асуудлыг үр ашигтай шийдэгддэг гэж үздэг явдал. Практикт бидэнд хэрэгтэй нь бидний санаа тавьж буй жишээнүүд (аж үйлдвэр, байгаль гэх мэтээс манай програм үүсгэдэг жишээнүүд) л манай алгоритмаар хурдан бодогдвол хангалттай бусад нь удаан байж болно. Магадгүй “төлөөлөх” (typical) жишээнүүд хурдан бодогдвол хангалттай ч байж мэднэ. Мэдээж, практикт ямар жишээнүүд гарч ирдгийг ойлгох нь өөрөө том асуудал бөгөөд төлөөлөх олон төрлийн загварууд, тэдэнд зориулагдсан алгоритмууд судлагддаг (бид үүнийг 4.4-р хэсэгт хөндөнө). Хамгийн муу тохиолдлын шинжилгээний илт давуу тал нь аль жишээ гарч ирэх талаар санаа зовох шаардлагагүй бүгдийг нь бидний “үр ашигтай алгоритм” хурдан бодно. Энэ ойлголт нь сайн нийлдэг (“композици” сайтай, жишээлбэл нэг алгоритм нөгөөгөө дэд програм болгон ашиглахад зохимжтой). Цаашлаад, орох оролт нь алгоритмыг удаашруулахыг хүсдэг үл мэдэгдэх өрсөлдөгчөөр үүсгэгддэг эсэргүү нөхцөл байдлыг тооцох боломж олгодог. Ийм өрсөлдөгчийг загварчлах нь криптограф, алдаа засварлалт зэрэг салбарт зайлшгүй бөгөөд үүнийг хамгийн муу тохиолдлын шинжилгээ хөнгөвчилдөг. Эцэст нь, дээр дурдсанчлан, энэ ойлголт нь нарийн төвөгтэй байдлын “орчлон”-ы маш гоёмсог бүтэцийг ил болгож, дундаж-тохиолдлын болон жишээ онцлог онолуудыг илүү нарийвчлан судлахад урам өгсөн.


P ангийг ойлгох нь маш их ач холбогдолтой. Онол, практикт маш олон тооцооллын асуудлууд үр ашигтай шийдэл шаарддаг. Сүүлийн хэдэн арван жилд олон төрлийн алгоритмын арга техникүүд хөгжиж, эдгээрийн ихээхнийг шийдэх боломж олгож байна (жишээ нь. \cite[CLR01, KT06] сурах бичгүүдийг үзэж болно). Эдгээр арга техник нь одоо бидний энгийн мэт ойлгодог гэрийн компьютерийн маш хурдан програмуудыг (вэб хайлт, үгийн алдаа шалгалт, өгөгдөл боловсруулах, компьютер тоглоомын график, хурдан арифметик) болон үйлдвэр, бизнес, математик, шинжлэх ухаанд хэрэглэгддэг илүү хүнд даацын програмуудыг хөдөлгөж байна. Гэвч (удахгүй бидний таарах) илүү өндөр практик, онолын үнэ цэнтэй олон асуудал хараахан тайлагдаагүй хэвээр. Үр ашигтайгаар бодогдох асуудлуудын анги P гэдэг энэ суурь математик объектыг тодорхойлж, онцлоглох сорилт нь одоогоор биднээс нэлээд алс байна.


Энэ хэсгийг бид янз бүрийн салбар дахь математик ач холбогдолтой, P ангид багтдаг ёс журамтай (nontrivial) хэдэн асуудлын жишээгээр өндөрлөе. Эдгээр алгоритмуудыг хөгжүүлэхэд шаардагдсан математик ба тооцооллын ойлголтуудын харилцан нөлөө илт харагдана. Ихэнх жишээ нь элементар шинжтэй; хэрэв танд танил бус математик ойлголт таарвал тухайн жишээг түр алгасаж болно (эсвэл бүр сайн нь, утгыг нь нягталж үзээрэй).


\begin{figure}[h]
  \centering
  \includegraphics[width=0.6\textwidth]{figs/3.png}
  \caption{Зүүн талд: төгс тохирол (perfect matching)-той граф (тохирол нь зургаар тэмдэглэсэн). Баруун талд: төгс тохиролгүй граф.}
  \label{fig:3-1}
\end{figure}


\subsection{P ангилалд багтах зарим бодлогууд}


\begin{itemize}
  \item \textbf{Төгс тааруулалт (Perfect Matching)}. Өгөгдсөн графд төгс тааруулалт байгаа эсэхийг шалгах, өөрөөр хэлбэл оройнуудыг хос хосоор нь тэгшитгэн, хос бүр нь тухайн графын нэг ирмэг байх боломжтой эсэхийг тодруулах (Зураг 3). Эдмондсын \cite{Edm65b} хэмээх уран нарийн алгоритм нь P ангилалд багтах “энгийн бус” анхны алгоритмуудын нэг байх магадлалтай бөгөөд дээр дурдсанчлан энэ өгүүлэл нь P ангиллыг судлах чухлыг онцлон харуулахад гол үүрэг гүйцэтгэсэн. Граф дахь тааруулалтын бүтэц нь комбинаторик дахь хамгийн их судлагдсан сэдвүүдийн нэг юм (жишээ нь, \cite{LP09}).
  \item \textbf{Энгийн тоо шалгах (primality testing).} Өгсөн бүхэл тоо анхны тоо эсэхийг тогтоох.\footnote{Жишээлбэл, $X = 6797727 × 2^15328$ үед $X − 1$ болон $X + 1$ хоёрын хувьд хариуг тодорхойлоод үз.} Гаусс энэ асуудалд үр ашигтай алгоритм олоход математикчдыг шууд уриалж байсан ч шийдэхэд хоёр зууны турш хэрэгтэй болсон. Агравал, Каял, Саксена нарын \cite{AKS04} саяхны энэ ололтын түүхийг Гранвилл cite{Gra05} сайхан хүүрнэсэн байдаг. Мэдээж, анхны тоонууд математик (болон поп соёл)-д ямар төв байр суурьтайг нурших шаардлагагүй.
  \item \textbf{Онгоцлог байхыг шалгах (planarity testing).} Өгөгдсөн граф онгоцлог уу, өөрөөр хэлбэл ирмэгүүд нь хоорондоо огт огтлолцохгүйгээр хавтгай дээр дүрсэлж болох уу? (Зураг 3 болон Зураг 5 дахь графуудад үүнийг туршаад үз.) Энэ суурь асуудлын шугаман хугацааны (linear time) алгоритмууд цуварч нээгдсэн бөгөөд анхны нэг нь Хопкрофт, Таржаны \cite{HT74} өгүүллээс эхэлдэг.
  \item \textbf{Шугаман програмчлал (linear programming).} Олон хувьсагчтай шугаман тэгшигдээгүй тэнцэтгэлцлүүдийн (жишээ Зураг 4-т бий) өгөгдмөл систем нь харилцан нийцтэй юу, өөрөөр хэлбэл бүх тэгшиглийг шийдийгг зэрэг хангах бодит утгууд хувьсагчдад олдож болох уу гэдгийг тогтоох. Энэ асуудал болон түүний оновчлолын хувилбар нь асар их хэрэгцээтэй. Энэ нь олон өөр асуудлыг (жишээ нь, тэг нийлбэрт тоглоомуудын оновч стратеги олох) хамардаг. Үүнийг үр ашигтайгаар бодох боломж олгодог овгор (convex) оновчлолын аргууд (\cite{Kha79}, \cite{Kar84}) нь үнэндээ үүнээс ч ихийг гүйцэтгэдэг (жишээ нь, Шрайвэрын ном \cite{Sch03}).
  \item \textbf{Олон гишүүнтийг задлах (factoring polynomials).} Рациональ (Q) коэффициенттой олон хувьсагчтай олон гишүүнтийг Q дээрх бууруулшгүй (ирредуцлагдашгүй) үржвэрүүдэд нь задлах. Ленстра, Ленстра, Ловаш нарын \cite{LLL82} бүтээсэн (гол нь $R^n$ орон зай дахь торын “богино суурь”-тай холбоотой) арга хэрэгслүүд нь олон бусад хэрэглээтэй (13.8-р хэсгийг үз).
  \item \textbf{Удамшлын графын шинжүүд (hereditary graph properties).}  Тогтмол өгөгдсөн минороор хаагдсан\footnote{Өөрөөр хэлбэл орой (эсвэл ирмэг) хасах, мөн ирмэгийг агшуулах (contract) үйлдлүүдийг хийсэн ч тухайн овог дотроо үлддэг.} (minor-closed) овгийн гишүүн эсэхийг төгсгөлтэй граф дээр шалгах. Ерөнхий тохиолдолд хугацаа нь маш өндөртэй ч полиноми хугацааны алгоритм нь Робертсон, Сеймур хоёрын ийм овгуудын асар том бүтцийн онол \cite{RS95}, түүнчлэн төгсгөлтэй суурийн теорем\footnote{Төгсгөлтэй суурийн теорем (finite basis theorem)}-ээс шууд дагалдана.
  \item \textbf{Ялгалтын бүлгийн гишүүнчлэл (permutation group membership).}  n элементийн хэд хэдэн ялгалт (пермутаци) өгөгдөхөд эхнийхийг нь бусад нь үүсгэж (generate) чадах уу?\footnote{Энд “бусад ялгалтуудын үүсгэсэн дэд бүлэгт эхний ялгалт ордог уу?” гэсэн гишүүнчлэлийн асуудлыг хэлж байна.} \cite{Sim70, FHL80}-д боловсруулсан коммутатив бус “Гауссын устгал”-ын техникүүд нь алгоритмын бүлэг судлалыг төрүүлж, бүлэг онолчдын өдөр тутмын хэрэгсэл болон, ялангуяа граф изоморфизм шалгалтын том дэвшилд хүргэсэн \cite{Bab15}.
  \item \textbf{Гиперболик бүлгийн үгийн асуудал (hyperbolic word problem).} Генератор ба харилцан хамаарлын (presentation) дурын өгөгдлөөр өгөгдсөн гиперболик бүлэг, мөн генераторуудаар бичигдсэн w гэсэн үг өгөгдөхөд w нь тэнцэт (identity) элементийг илэрхийлэх эсэхийг шалгах. Громовын геометрийн арга барил, тэр дундаа ийм бүлгүүдийн Кейли графууд дахь изопериметрийн хязгаарууд \cite{Gro87} нь полиноми хугацааны (цаашлаад түүнээс ч илүү) алгоритм олгоно. Харин ерөнхий төгсгөлийн тоогоор өгөгдсөн бүлгүүдийн хувьд энэ асуудал шийдвэрлэшгүй.
\end{itemize}


\begin{figure}[h]
  \centering
  \includegraphics[width=0.6\textwidth]{figs/4.png}
  \caption{Хоёр хувьсагч, гурван тэгш бус байдал бүхий хоёр шугаман програм}
  \label{fig:3-2}
\end{figure}


\section{Үр ашигтай баталгаажуулалт ба NP ангилал}


$C \subset I$ нь ангиллын асуудал (classification problem) байг.\footnote{Компьютерийн ухааны бичвэрд С нь ихэвчлэн хэл гэж бичигддэг} Тодруулбал, оролтод математик объектыг дүрслэх хоёртын дараалал $x \in I$ өгөгдөнө, бидний зорилт $x \in C$ эсэхийг тогтоох. (C)-г “шинж” гэж үзэхэд эвтэйхэн:  $x \in C$ бол тухайн шинжийг агуулсан объектууд,  $x \notin C$ бол агуулдаггүй объектууд. Хэрэв бидэнд (C)-д зориулсан үр ашигтай алгоритм байвал түүнийг (x)-д хэрэглэхэд л (x) нь (C) шинжтэй эсэх нь тодорно. Харин тийм алгоритм үгүй бол “дараагийн шилдэг хувилбар” нь юу вэ? Нэг хариулт нь $x \in C$ гэдгийг батлах хэрэгтэй. Формалиар тодорхойлохоосоо өмнө хоёр сонирхолтой жишээ үзье.


Эхний жишээ бол 1903 онд Америкийн Математикийн Нийгэмлэгийн хурлаар Ф. Н. Колегийн “Том тоонуудыг задлах нь” хэмээх лекцийн алдарт түүх юм. Тэр ганц ч үг дуугаралгүй самбар дээр: \\
$2^67 - 1 = 147573952589676412927 = 193707721761838257287 $\\
эж бичээд баруун талын хоёр тоог уртаар үржүүлж зүүн талын тоо Мерсеннийн 67 дахь тоо (анхны тоо байж магадгүй гэж таамаглагдаж байсан)-г гарган үзүүлжээ. Танхимд асуулт тавих хүн байсангүй.


Энд юу болов? Коле $2^67-1$ нийлмэл тоо гэдгийг үзүүлсэн. Үнэхээр, $x \in COMPOSITES$  (нийлмэл тоонуудын олонлог) хэлбэрийн (зөв) ямар ч мэдэгдэлд маш богино нотолгоо өгч болдог: зүгээр л (x)-ийн үржвэр задлалыг өгнө. Энэ жишээнээс бид хоёр гол шинжийг ялгаж авна: энэ нотолгоо богино ба амархан шалгагдана. Үржүүлэгчдийн нийт урт нь оролтын урттай ойролцоо, харин тэдгээрийг хооронд нь үржүүлэх үр ашигтай алгоритм бий. Колед эдгээр үржүүлэгчдийг олох нь хэчнээн бэрх (түүний хэлснээр “гурван жилийн ням гараг”-ийг зарцуулсан) байсан нь уг нотолгоог үзүүлэхэд огт нөлөөлөөгүйг анзаар.



Хоёр дахь жишээ бол математикчдын өдөр тутамд хийдэг зүйл бол сэтгүүлийн өгүүлэл унших. Тэнд ихэвчлэн (түүнчлэн нотолсон гэж) нэг теорем дагалдсан нотолгоогоор дүүрэн. Тэгэхээр бид $x \in THEOREMS$ маягийн мэдэгдлүүдийг баталгаажуулж байна, энд $x \in THEOREMS$ нь, жишээ нь, олонлогын онолын хүрээнд нотлогдох бүх мэдэгдлийн олонлог. Бичигдсэн нотолгоо богино (хуудасны хязгаар байдаг) ба амархан шалгагдана (рецензент “боломжийн” хугацаанд нягталж чадна) гэж угтаа таамагладаг. Үүнийг формалаар бичих боломжтой. Бас дахин тэмдэглэхэд, зохиогчид нотолгоог олоход хэр удаан суудаг нь бидэнд падгүй. Мэдээж, сэтгүүлийн теорем/нотолгоонууд бүрэн албаны формаль хэл дээр бичигдээгүй; реценз хийх ажлыг эдгээр “хагас-формаль” нотолгоог формаль болгож, мэдэгдлийн үнэн чанарыг тогтоох боломжтой эсэхийг нягтлахтай дүйцнэ гэж ойлгож болно.


Одоо бид дээрх хоёр жишээг бүхэлд нь хамарсан асуудлуудын ангилал NP-ийг тодорхойлоход бэлэн боллоо.


NP ангилал гэдэг нь, гишүүнчлэл (өөрөөр хэлбэл $x \in C$ хэлбэрийн мэдэгдэл) нь богино, үр ашигтайгаар шалгагддаг нотолгоотой бүх шинж $C$-г хамарна. Урьдын адил “полином”-оор аль аль ойлголтыг тогтооно. $x \in C$  гэдгийг батлах болзошгүй нотолгоо $y$-ийн урт нь $x$-ийн уртын полиномоос хэтрэхгүй байх ёстой. Мөн өгөгдсөн $y$ үнэхээр $x \in C$-г нотолж буй эсэхийг шалгах явц нь полиноми хугацаанд хийгдэх (энэ шалгагч алгоритмыг $V_C$ гэе). Эцэст нь, хэрэв $x \notin C$ бол тийм $y$ огт байхгүй байх ёстой. Үүнийг формалаар бичье.


\begin{definition}
  (NP анги.) Олонлог $C$ нь NP ангилалд багтана, хэрэв $V_C \in \textbf{P}$ функц ба тогтмол (k) оршин тогтнож:
  \begin{itemize}
    \item Хэрвээ $x \in C$ бол $\exists _y$ $|y| \leq k \cdot |x|^k$ $V_C(x,y)=1$ байна.
    \item Хэрвээ $x \notin C$ бол $\forall _y$ байхад $V_C(x,y)=0$ байна.
  \end{itemize}
\end{definition}


Логикийн өнцгөөс, NP дэх олонлог бүр $C$ нь шалгалтын процесс $V_C$-ээр тодорхойлогдсон бүрэн ба үнэн зөв complete ба sound нотолгооны системийн теоремүүдийн олонлог гэж үзэгдэж болно.


$V_C$-г “итгүүлэх” дараалал $y$-г ихэвчлэн тухайн $x \in C$-ийг гэрч (witness) эсвэл гэрчилгээ (certificate) гэдэг. Ахин тэмдэглэхэд, NP-гийн тодорхойлолт нь $y$-г олох хэчнээн хэцүү эсэхэд бус, $y$-г ашиглан $x \in C$-г үр ашигтай шалгаж чаддагаараа л хамаатай. Тийм $y$ (хэрэв байдаг бол) нь бүхнийг чадагч оршихуй өгсөн мэт, эсвэл зүгээр “таасан” мэт харагдаж болно. Үнэндээ NP гэдэг товчилбор нь “Nondeterministic Polynomial time” буюу тодорхой бус (нондетерминистик) машинаар “гэрч $y$-г” (хэрэв оршин байвал) тааж, дараа нь түүнийг детерминистик байдлаар шалгах чадварыг илэрхийлнэ.


Гэсэн ч гэрч олох хүндрэл нь мэдээж чухал. Энэ нь NP олонлогуудтай холбоотой хайлтын асуудлыг тодорхойлдог. NP дахь аливаа шийдвэрийн асуудал $C$ (цаашлаад $C$-д хамаарах аливаа шалгагч $V_C$) нь түүнд натурал нэг хайлтын асуудлыг тодорхойлно: өгөгдсөн $x \in C$-д $V_C$-г “итгүүлэх” богино гэрч $y$-г олно. Энэ хайлтын асуудлын зөв шийдлийг $V_C$ үр ашигтайгаар баталгаажуулна гэхдээ тодорхойлолтоороо тийм.


Гэрчийг (хэрэв оршин байвал) “бүрэн хайлтаар” (brute force) олж болно: гэрчүүд богино $(\text{poly}(n)$ урттай) тул боломжит бүх $y$-г тоолж, бүр дээр нь шалгалтыг ажиллуулна. Гэхдээ энэ тооллого нь $n$-д экспоненциал хугацаа шаарддаг. Энэ бүлгийн (цаашлаад энэхүү ном, бүр тооцооллын онолын!) гол асуулт бол NP дэх бүх асуудлуудад ийм дан brute force бодитойгоор (полиноми хугацаанд) хурдан болгох алгоритмууд байдаг эсэх явдал юм.


Ихэвчлэн хайлтын хувилбарууд натуралаа илүү “төрдөг” ч, практикт тэдгээрийн шийдвэрийн хувилбарыг (өөрөөр хэлбэл богино гэрч байгаа эсэх-ийг) судлах нь эвтэйхэн байдаг. Бараг бүх тохиолдолд шийдвэр ба хайлтын хувилбарууд тооцооллын хувьд эквивалент.\footnote{Онцгой тохиолдол байж болох нэг жишээ нь $COMPOSITES$ буюу нийлмэл тоонуудын олонлог бөгөөд түүнд зориулсан баталгаажуулах процедур нь хүчингүй бус (nontrivial) үржвэрийн хүчин зүйл-ийг гэрч болгон хүлээн зөвшөөрдөг.
Анхаарах хэрэгтэй зүйл нь, $COMPOSITES \in P$ нь шийдвэрлэх (decision) асуудал боловч, түүнтэй холбоотой хайлт (search) асуудал нь бүхэл тооны задлал (integer factorization)-тай эквивалент бөгөөд энэ нь одоогоор үр ашигтай алгоритмтай гэдэг нь мэдэгдээгүй юм.}


Дараах нь NP дэх зарим асуудлууд (нарийвчилбал шинжүүд) бөгөөд дээр дурдсан (\textsc{THEOREMS}), (\textsc{COMPOSITES})-оос гадна оршино. Зарим нь өмнөх P ангиллын жишээний хувилбарууд. Гэвч эдгээрийн алиных нь ч P-д багтах эсэхийг бид огт мэдэхгүй. Уншигч танд дасгал болгож (ихэнх нь амархан, зарим нь тийм биш) тус бүрийн хувьд тухайн шинжийг агуулсан оролтуудад “богино, амархан шалгагдах” гэрчийг тодорхойлж үзээрэй.\footnote{Хэцүү онцгой тохиолдол нь Matrix Group Membership буюу матрицын бүлэгт гишүүн эсэхийг шалгах асуудал юм. Хэрвээ та үүнийг өөрөө шийдэж чадахгүй бол, сайхан бичигдсэн \cite{BS84} бүтээлийг нэг хараад үзээрэй.}


NP-дэх зарим асуудлууд:
\begin{itemize}
\end{itemize}
