\documentclass[11pt]{report}

\usepackage{fontspec}
\setmainfont{Times New Roman}

\usepackage[a4paper,margin=1in]{geometry}
\usepackage{microtype}
\usepackage{graphicx}
\usepackage{hyperref}
\hypersetup{
  unicode=true,
  colorlinks=true,
  linkcolor=blue,
  urlcolor=blue,
  citecolor=blue
}


\usepackage{amsthm}
\theoremstyle{plain}
\newtheorem{theorem}{Теорем}[chapter]
\newtheorem{lemma}[theorem]{Лемма}
\newtheorem{corollary}[theorem]{Дүгнэлт}

\theoremstyle{definition}
\newtheorem{definition}[theorem]{Тодорхойлолт}
\newtheorem{example}[theorem]{Жишээ}

\theoremstyle{remark}
\newtheorem{remark}[theorem]{Тайлбар}
\renewcommand{\proofname}{Нотолгоо}
\renewcommand{\qedsymbol}{$\square$}

\usepackage[nameinlink]{cleveref}
\crefname{theorem}{теорем}{теоремууд}
\Crefname{theorem}{Теорем}{Теоремууд}
\crefname{lemma}{лемма}{лемманууд}
\Crefname{lemma}{Лемма}{Лемманууд}
\crefname{corollary}{дүгнэлт}{дүгнэлтүүд}
\Crefname{corollary}{Дүгнэлт}{Дүгнэлтүүд}
\crefname{definition}{тодорхойлолт}{тодорхойлолтууд}
\Crefname{definition}{Тодорхойлолт}{Тодорхойлолтууд}
\crefname{remark}{тайлбар}{тайлбарууд}
\Crefname{remark}{Тайлбар}{Тайлбарууд}

\usepackage{titlesec}
\titleformat{\chapter}[display]
 {\Huge\bfseries}
  {}
  {0pt}
  {}
\usepackage{biblatex}
\addbibresource{biblio.bib}


\title{Computation Theory}
\author{B221910003 Б.Төгөлдөр}
\date{\today}

\begin{document}
\maketitle
\tableofcontents
\chapter{Бүлэг 1}
\label{chap:regular}

\chapter{Бүлэг 2}
\label{chap:regular}


\chapter{Бүлэг 3. Тооцооллын хүндрэл 101: Суурь ойлголтууд, P ба NP}
\label{chap:regular}

Энэ бүлэгт бид тооцооллын асуудлуудын үндсэн ойлголт, өгөгдлийн төлөөлөл, үр ашигтай тооцоолол, асуудлуудын хоорондын үр ашигтай бууруулалт (reduction), нотолгоог үр ашигтайгаар шалгах, P, NP, coNP ангиуд, мөн NP-бүрэн (NP-complete) асуудлын ойлголтыг авч үзнэ. Бид голчлон ангиллын буюу шийдвэрийн (decision) асуудлууд дээр төвлөрнө ба өөр төрлийн асуудлууд ба ангиуд (тооллого, ойролцоолол, байгуулах зэрэг) ч мөн судлагддаг бөгөөд заримыг нь энэ номын дараагийн хэсгүүдэд хэлэлцэнэ.

\section{Сэдэлжүүлэх жишээнүүд}
Дараах гурван шийдвэрийн ангиллын асуудлыг авч үзье. 2-р бүлэгийн адил, ийм төрлийн ангиллын асуудал бүрт бидэнд нэг объектын тайлбар өгөгдөнө, тэгээд тэр нь хүссэн шинжийг агуулж байгаа эсэхийг шийдэх ёстой болно.

\begin{enumerate}
  \item $Ax^2+By+C=0$ хэлбэрийн ямар Диофантын тэгшитгэлүүдийг эерэг бүхэл тооноор шийдэж болох вэ?
  \item 3 хэмжээст маннифолдууд дахь ямар “зангилаа” (knots) нь төрөл $\le g$ бүхий гадаргуугаар хязгаарлагдах вэ?
  \item Ямар хавтгай газрын зураг (планар граф) 3 өнгөөр будагдах боломжтой вэ?
\end{enumerate}

Асуудал (1′) нь 2-р бүлгийн (1) асуудлын хязгаарлалт юм. Асуудал (1) нь шийдэшгүй (undecidable) байсан тул Диофантын тэгшитгэлийн илүү хязгаарлагдсан ангиллуудыг илүү сайн ойлгох нь зүй ёсны хэрэг. Асуудал (2′) нь 2-р бүлгийн (2) асуудлын хоёр талаараа ерөнхийлөлт юм: unknotting асуудал (2) нь маннифолд $R^3$ ба төрөл g=0-ийн тусгай тохиолдлыг авч үздэг. Асуудал (2) нь шийдэгддэг (decidable) байсан бөгөөд түүний ерөнхийлөлт (2′) мөн шийдэгддэг эсэхийг мэдэхийг хүсэж болох юм. Асуудал (3′) нь (3) асуудлын сонирхолтой хувилбар. Бүх газрын зураг 4 өнгөөр будагддаг боловч бүх зураг 3 өнгөөр будагддаггүй; зарим нь чадна, зарим нь чаддаггүй. Тиймээс энэ нь ойлгоход бэрхшээлтэй өөр нэгэн ангиллын асуудал юм.
Ихэнх математикчид эдгээр гурав нь хоорондоо огт холбоогүй асуудлууд гэж үзэх хандлагатай: тус бүр нь өөр өөр чиглэл—тус тусдаа ойлголт, зорилго, хэрэгслүүдтэй альгебр, топологи, комбинаторик харьяалагдана. Гэвч доорх теорем энэ үзлийг буруу байж магадгүйг санал болгож байна.


\begin{theorem}
  (1`), (2`), (3`) асуудлууд нь эквивалент.
\end{theorem}


\chapter{Бүлэг 4. NP-д болон түүний ойролцоох асуудлууд ба ангиллууд}
\label{chap:regular}



\printbibliography

\end{document}

